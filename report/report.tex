%%%%%%%%%%%%%%%%%%%%%%%%%%%%%%%%%%%%%%%%%
% Programming/Coding Assignment
% LaTeX Template
%
% This template has been downloaded from:
% http://www.latextemplates.com
%
% Original author:
% Ted Pavlic (http://www.tedpavlic.com)
%
% Note:
% The \lipsum[#] commands throughout this template generate dummy text
% to fill the template out. These commands should all be removed when 
% writing assignment content.
%
% This template uses a Perl script as an example snippet of code, most other
% languages are also usable. Configure them in the "CODE INCLUSION 
% CONFIGURATION" section.
%
%%%%%%%%%%%%%%%%%%%%%%%%%%%%%%%%%%%%%%%%%

%----------------------------------------------------------------------------------------
%	PACKAGES AND OTHER DOCUMENT CONFIGURATIONS
%----------------------------------------------------------------------------------------

\documentclass{article}

\usepackage{fancyhdr} % Required for custom headers
\usepackage{lastpage} % Required to determine the last page for the footer
\usepackage{extramarks} % Required for headers and footers
\usepackage[usenames,dvipsnames]{color} % Required for custom colors
\usepackage{graphicx} % Required to insert images
\usepackage{listings} % Required for insertion of code
\usepackage{courier} % Required for the courier font
\usepackage{lipsum} % Used for inserting dummy 'Lorem ipsum' text into the template
\usepackage[dutch]{babel}
\usepackage{amsmath,amssymb}

\newtheorem{stelling}{Stelling}
\newenvironment{proof}{\paragraph{Bewijs: }}{\hfill$\square$}


% Margins
\topmargin=-0.45in
\evensidemargin=0in
\oddsidemargin=0in
\textwidth=6.5in
\textheight=9.0in
\headsep=0.25in

\linespread{1.1} % Line spacing

% Set up the header and footer
\pagestyle{fancy}
\lhead{\hmwkAuthorName, \hmwkAuthorNameS} % Top left header
\rhead{ \hmwkClassInstructor\ \hmwkClassTime \hmwkTitle} % Top center head
%\rhead{\firstxmark} % Top right header
\lfoot{\lastxmark} % Bottom left footer
\cfoot{} % Bottom center footer
\rfoot{Page\ \thepage\ of\ \protect\pageref{LastPage}} % Bottom right footer
\renewcommand\headrulewidth{0.4pt} % Size of the header rule
\renewcommand\footrulewidth{0.4pt} % Size of the footer rule

\setlength\parindent{0pt} % Removes all indentation from paragraphs

%----------------------------------------------------------------------------------------
%	CODE INCLUSION CONFIGURATION
%----------------------------------------------------------------------------------------

\definecolor{MyDarkGreen}{rgb}{0.0,0.4,0.0} % This is the color used for comments
\lstloadlanguages{Perl} % Load Perl syntax for listings, for a list of other languages supported see: ftp://ftp.tex.ac.uk/tex-archive/macros/latex/contrib/listings/listings.pdf
\lstset{language=Perl, % Use Perl in this example
        frame=single, % Single frame around code
        basicstyle=\small\ttfamily, % Use small true type font
        keywordstyle=[1]\color{Blue}\bf, % Perl functions bold and blue
        keywordstyle=[2]\color{Purple}, % Perl function arguments purple
        keywordstyle=[3]\color{Blue}\underbar, % Custom functions underlined and blue
        identifierstyle=, % Nothing special about identifiers                                         
        commentstyle=\usefont{T1}{pcr}{m}{sl}\color{MyDarkGreen}\small, % Comments small dark green courier font
        stringstyle=\color{Purple}, % Strings are purple
        showstringspaces=false, % Don't put marks in string spaces
        tabsize=5, % 5 spaces per tab
        %
        % Put standard Perl functions not included in the default language here
        morekeywords={rand},
        %
        % Put Perl function parameters here
        morekeywords=[2]{on, off, interp},
        %
        % Put user defined functions here
        morekeywords=[3]{test},
       	%
        morecomment=[l][\color{Blue}]{...}, % Line continuation (...) like blue comment
        numbers=left, % Line numbers on left
        firstnumber=1, % Line numbers start with line 1
        numberstyle=\tiny\color{Blue}, % Line numbers are blue and small
        stepnumber=5 % Line numbers go in steps of 5
}

% Creates a new command to include a perl script, the first parameter is the filename of the script (without .pl), the second parameter is the caption
\newcommand{\perlscript}[2]{
\begin{itemize}
\item[]\lstinputlisting[caption=#2,label=#1]{#1.pl}
\end{itemize}
}

%----------------------------------------------------------------------------------------
%	DOCUMENT STRUCTURE COMMANDS
%	Skip this unless you know what you're doing
%----------------------------------------------------------------------------------------

% Header and footer for when a page split occurs within a problem environment
\newcommand{\enterProblemHeader}[1]{
\nobreak\extramarks{#1}{#1 continued on next page\ldots}\nobreak
\nobreak\extramarks{#1 (continued)}{#1 continued on next page\ldots}\nobreak
}

% Header and footer for when a page split occurs between problem environments
\newcommand{\exitProblemHeader}[1]{
\nobreak\extramarks{#1 (continued)}{#1 continued on next page\ldots}\nobreak
\nobreak\extramarks{#1}{}\nobreak
}

%\setcounter{secnumdepth}{0} % Removes default section numbers
%\newcounter{homeworkProblemCounter} % Creates a counter to keep track of the number of problems

%\newcommand{\homeworkProblemName}{}
%\newenvironment{homeworkProblem}[1][Problem \arabic{homeworkProblemCounter}]{ % Makes a new environment called homeworkProblem which takes 1 argument (custom name) but the default is "Problem #"
%\stepcounter{homeworkProblemCounter} % Increase counter for number of problems
%\renewcommand{\homeworkProblemName}{#1} % Assign \homeworkProblemName the name of the problem
%\section{\homeworkProblemName} % Make a section in the document with the custom problem count
%\enterProblemHeader{\homeworkProblemName} % Header and footer within the environment
%}{
%\exitProblemHeader{\homeworkProblemName} % Header and footer after the environment
%}

\newcommand{\problemAnswer}[1]{ % Defines the problem answer command with the content as the only argument
\noindent\framebox[\columnwidth][c]{\begin{minipage}{0.98\columnwidth}#1\end{minipage}} % Makes the box around the problem answer and puts the content inside
}

%\newcommand{\homeworkSectionName}{}
%\newenvironment{homeworkSection}[1]{ % New environment for sections within homework problems, takes 1 argument - the name of the section
%\renewcommand{\homeworkSectionName}{#1} % Assign \homeworkSectionName to the name of the section from the environment argument
%\subsection{\homeworkSectionName} % Make a subsection with the custom name of the subsection
%\enterProblemHeader{\homeworkProblemName\ [\homeworkSectionName]} % Header and footer within the environment
%}{
%\enterProblemHeader{\homeworkProblemName} % Header and footer after the environment
%}

%----------------------------------------------------------------------------------------
%	NAME AND CLASS SECTION
%----------------------------------------------------------------------------------------

\newcommand{\hmwkTitle}{Practicum Numerieke Wiskunde} % Assignment title
\newcommand{\hmwkSubject}{Benadering van functies door veeltermen} % Due date
\newcommand{\hmwkClass}{COMPSCI\ 101} % Course/class
\newcommand{\hmwkClassTime}{ } % Class/lecture time
\newcommand{\hmwkClassInstructor}{ } % Teacher/lecturer
\newcommand{\hmwkAuthorName}{Sarah Crombez} % Your name
\newcommand{\hmwkAuthorNameS}{Zimcke Van de Staey}

%----------------------------------------------------------------------------------------
%	TITLE PAGE
%----------------------------------------------------------------------------------------

\title{
\vspace{2in}
\textmd{\textbf{ \hmwkTitle}}\\
\vspace{0.1in}\normalsize\vspace{0.1in} \large{\hmwkSubject}\\
%\vspace{0.1in}\large{\textit{\hmwkClassInstructor\ \hmwkClassTime}}
\vspace{3in}
}

\author{\textbf{\hmwkAuthorName}\ \\ \textbf{\hmwkAuthorNameS}\ }
\date{} % Insert date here if you want it to appear below your name

%----------------------------------------------------------------------------------------

\begin{document}

\maketitle

%----------------------------------------------------------------------------------------
%	TABLE OF CONTENTS
%----------------------------------------------------------------------------------------

%\setcounter{tocdepth}{1} % Uncomment this line if you don't want subsections listed in the ToC

\newpage
\tableofcontents
\newpage

%----------------------------------------------------------------------------------------
%	PART 0
%----------------------------------------------------------------------------------------

% To have just one problem per page, simply put a \clearpage after each problem

%\begin{homeworkProblem}
\section{Achtergrond}

De Chebyshev veeltermen van de eerst soort $T_{k}(x)$ worden gedefinieerd op basis van de volgende recursiebetrekking:
\begin{center}
$T_{0}(x)=1$\\
$T_{1}(x)=x$\\
$T_{k+1}(x)=2xT_{k}(x)-T_{k-1}(x)$
\end{center}

\begin{stelling}
Op het interval $[-1,1]$ voldoen de Chebyshev veeltermen aan volgende vergelijking:
\begin{equation}
T_{k}(x)=\cos(k\arccos(x)) \label{formule1}
\end{equation}
\end{stelling}

\problemAnswer{
\begin{proof}
We zullen deze stelling aantonen met behulp van volledige inductie.\\

\begin{bfseries} Basisstap:\end{bfseries}\\
voor $k=0$ geldt: $T_{0}=\cos(0*\arccos(x))=\cos(0)=1$\\
voor $k=1$ geldt: $T_{1}=\cos(\arccos(x))=x$ op het interval $[-1,1]$ want de $\arccos$-functie is enkel 	gedefinieerd op het interval $[-1,1]$.\\ \\
\vspace{0.1in}
\begin{bfseries} Inductiestap:\end{bfseries}\\
We nemen aan dat voor alle $j\leq n$ geldt: $T_{n}(x)=\cos(n\arccos(x))$. Nu moet aangetoond worden dat dit ook 	geldt voor $n+1$. Volgens de recursiebetrekking voor de Chebyshev veeltermen geldt: $T_{n+1}(x)=2xT_{n}(x)-T_{n-1}(x)$. Nu kunnen we de inductiehypothese toepassen, dit geeft: $T_{n+1}(x)=2x\cos(n\arccos(x))-\cos((n-1)\arccos(x))$. Met behulp van de som-en verschilformules voor de cosinus kunnen we dit schrijven als: $T_{n+1}=2x\cos(n\arccos(x))-\cos(n\arccos(x))\cos(\arccos(x))-\sin(n\arccos(x))\sin(\arccos(x))\\
=\cos(n\arccos(x)\cos(\arccos(x))-\sin(n\arccos(x))\sin(\arccos(x))$
Hierop kunnen we dan opnieuw de som-en verschil formules voor de cosinus op toepassen en dit geeft:
$T_{n+1}(x)=\cos((n+1)\arccos(x))$. De veronderstelling geldt dus ook voor n+1.\\ \\
\vspace{0.1in}
\begin{bfseries} Conclusie:\end{bfseries}\\
Uit de basisstap, de inductiestap en het principe van volledige inductie volgt het te bewijzen.
\end{proof}
}

\clearpage
%Listing \ref{homework_example} shows a Perl script.
%\perlscript{homework_example}{Sample Perl Script With Highlighting}
%\end{homeworkProblem}

%----------------------------------------------------------------------------------------
%	PART 1
%----------------------------------------------------------------------------------------

%\begin{homeworkProblem}

\section{Deel 1: Drie veeltermbasissen}
\problemAnswer{

}
\clearpage
%\end{homeworkProblem}

%----------------------------------------------------------------------------------------
%	PART 2
%----------------------------------------------------------------------------------------

\section{Deel 2: Veelterminterpolatie}

\subsection{Equidistante punten en het Runge fenomeen}
\begin{center}
\begin{figure}
\includegraphics[width=0.75\columnwidth]{lagrange_interpolatie}
\caption{De Runge functie en de betreffende Lagrange interpolaties} % Example image
\end{figure}
\end{center}

\begin{center}
\begin{figure}
\includegraphics[width=0.75\columnwidth]{relatieve_fout}
\caption{De relatieve fout behorende bij de betreffende veelterminterpolaties} % Example image
\end{figure}
\end{center}

\subsection{Verschillende basissen}


\begin{center}
\begin{figure}
\includegraphics[width=0.75\columnwidth]{verschillende_basissen}
\caption{De Runge functie en de betreffende Lagrange interpolaties} % Example image
\end{figure}
\end{center}

\clearpage

%----------------------------------------------------------------------------------------
%	PART 3
%----------------------------------------------------------------------------------------

\section{Deel 3: Methode van Newton-Raphson}

\clearpage

%----------------------------------------------------------------------------------------
\end{document}